\chapter{Zaključak i budući rad}
				
		 
		 Zadatak našeg tima bio je razviti web-aplikaciju za stručne konferencije koja bi organizatorima omogućila veći doseg ljudi budući da bi se na konferenciji moglo sudjelovati i online te bi se u sklopu svake konferencije mogli pregledavati stručni radovi (posteri) drugih ljudi i za te radove glasati. Na projektu smo radili 3 mjeseca i naposljetku ga uspješno završili i ostvarili sve potrebne funkcionalnosti.\\
		 
		 U prvoj fazi projekta, prije samog početka izrade projektnog zadatka, diskutirali smo o njegovoj temi i razmišljali kako najbolje organizirati i razviti aplikaciju. Nakon uspostave plana, izradili smo bazu podataka koja je ključni dio svake aplikacije, krenuli s izradom dokumentacije pa naposljetku i sa samom implementacijom. U ovoj smo fazi implementirali većinu potrebnih funkcionalnost poput prijave, registracije, stvaranja konferencije i admina. Pojedini su članovi imali nešto više programerskog iskustva što je zasigurno pomoglo bržem napretku projekta i drugim članovima tima dalo dovoljno vremena da se uhodaju u svoje zadatke. Prije predaje prve revizije aplikaciju smo postavili na Render što nam je bio jedan od brojnih tehničkih izazova ove faze.

		 U drugoj smo fazi implementirali ostale potrebne funkcionalnosti među kojima su dodavanje i prikaz postera, fotografija i promotivnih materijala. U toj smo se fazi susreli s izazovima oko prikazivanja fotografija, no i taj smo problem riješili i stekli potrebna znanja za ubuduće. Dorađivanjem finesa, testiranjem aplikacije i dovršetkom dokumentacije završili smo sa svojim projektnim zadatkom vrlo zadovoljni učinjenim.
		 
		 Za cijelo vrijeme trajanja izrade projekta, vrlo se bitnom stavkom pokazala međusobna komunikacija i pomoć među članovima tima, redoviti online sastanci i zajedništvo u rješavanju izazova te uključenost našeg voditelja u sve korake procesa.
		 
		 Moguća nadogradnja aplikacije donijela bi korisnicima mogućnost primanja obavijesti o događanjima na konferenciji i davanje feedbacka na cjelokupnu konferenciju, a tijekom vremena zasigurno bi se moglo nadodati još novih funkcionalnosti koje bi korisnicima poboljšale iskustvo prisustvovanja na konferenciji.\\
		 
		 Sudjelovanje u ovom projektu bilo je intenzivno iskustvo koje nam je pružilo dublji uvid u kompleksnosti timskog rada programera. Razvili smo vještine organizacije, suradnje i komunikacije unutar tima. Rad na projektu pružio nam je priliku primijeniti teorijsko znanje u stvarnom okruženju praktičnom primjenom principa programskog inženjerstva. Shvatili smo važnost timskog duha, prilagodbe te jasne komunikacije, a istovremeno smo svjesni prostora za osobni i timski napredak. Za brže i kvalitetnije izvođenje projekata u budućnosti, prepoznajemo potrebu za usavršavanjem u područjima agilnog upravljanja projektima, testiranja softvera te optimizacije koda. Ovo iskustvo postat će temelj našeg profesionalnog razvoja, osiguravajući da budući projekti budu izvedeni s još većim uspjehom.
		 
		 
		 
		 
		 
		 
		
		 
		  
		 
		 
		 
		 
		
		\eject 