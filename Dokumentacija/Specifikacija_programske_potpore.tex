\chapter{Specifikacija programske potpore}
		
	\section{Funkcionalni zahtjevi}


			
			\noindent \textbf{Dionici:}
			
			\begin{packed_enum}
				
				\item Primarni dionici:
				
				\begin{packed_enum}
					\item Naručitelj
					\item Razvojni tim
					\item Rukovoditelji razvoja
					\item Strateški klijenti	
				\end{packed_enum}
				
				\item Sekundarni dionici:	
				
				\begin{packed_enum}
					\item Sudionici konferencije
					\item Zaposlenici konferencije
					\item Autori	
				\end{packed_enum}		
				
			\end{packed_enum}
			
			\noindent \textbf{Aktori i njihovi funkcionalni zahtjevi:}
			
			
			\begin{packed_enum}
				\item Neregistrirani posjetitelji konferencije
				
			\begin{quote}
				Neregistrirani posjetitelji konferencije imaju mogućnost pregledavanja stručnih postera svih prijavljenih sudionika pomoću lozinke dobivene na konferenciji i dodatna registracija u sustav po želji.
			\end{quote}
				
				\item Registrirani posjetitelji konferencije
				
		\begin{quote}
			Registrirani posjetitelji konferencije imaju dodatnu mogućnost glasanja za najbolji stručni poster uključujući pristup rezultatima glasovanja kako bi vidjeli koji poster osvaja najviše glasova kao i uživo praćenje trenutnih događanja u glavnoj konferencijskoj dvorani. Također imaju priliku vidjeti promotivne materijale pokrovitelja konferencije te podatke o trenutnim vremenskim uvjetima i vremenskoj prognozi za navedenu lokaciju. Po završetku, registrirani posjetitelji konferencije će moći pregledavati i preuzimati objavljene fotografije konferencije.
		\end{quote}
		
				\item Sistemski administrator
				\begin{quote}
					Sistemski administrator obavlja prijavu autora, radova i postera te ima mogućnost biranja fotografija sa konferencije kao i definiranja svih potrebnih uvjeta za rad sustava.
				\end{quote}
				
				\item Autori
				\begin{quote}
					Autori su glavni sudionici konferencije koji će nastupati. Oni prije početka konferencije elektroničkom poštom dostavljaju sve potrebne materijale (postere) sistemskom administratoru. Također, nakon završetak konferencije elektroničkom poštom dobivaju informaciju o njihovom rangu kao i pozivnicu za dodjelu nagrada.
				\end{quote}
				
				\item Pokrovitelji
				\begin{quote}
					Pokrovitelji konferenciju financiraju samu konferenciju i dostavljaju promotivne materijale sistemskom administratoru koji će se prikazivati registriranim posjetiteljima
				\end{quote}
				
				
				\item \textit{Aplikacija i baza podataka}
				\begin{quote}
					\textit{Pasivni faktori same funkcionalnosti aplikacije kao i spremanja podataka}
				\end{quote}
			\end{packed_enum}
			
			\eject 
			
			
				
			\subsection{Obrasci uporabe}
				
				\textbf{\textit{dio 1. revizije}}
				
				\subsubsection{Opis obrazaca uporabe}
					\textit{Funkcionalne zahtjeve razraditi u obliku obrazaca uporabe. Svaki obrazac je potrebno razraditi prema donjem predlošku. Ukoliko u nekom koraku može doći do odstupanja, potrebno je to odstupanje opisati i po mogućnosti ponuditi rješenje kojim bi se tijek obrasca vratio na osnovni tijek.}\\
					

					\noindent \underbar{\textbf{UC$<$broj obrasca$>$ -$<$ime obrasca$>$}}
					\begin{packed_item}
	
						\item \textbf{Glavni sudionik: }$<$sudionik$>$
						\item  \textbf{Cilj:} $<$cilj$>$
						\item  \textbf{Sudionici:} $<$sudionici$>$
						\item  \textbf{Preduvjet:} $<$preduvjet$>$
						\item  \textbf{Opis osnovnog tijeka:}
						
						\item[] \begin{packed_enum}
	
							\item $<$opis korak jedan$>$
							\item $<$opis korak dva$>$
							\item $<$opis korak tri$>$
							\item $<$opis korak četiri$>$
							\item $<$opis korak pet$>$
						\end{packed_enum}
						
						\item  \textbf{Opis mogućih odstupanja:}
						
						\item[] \begin{packed_item}
	
							\item[2.a] $<$opis mogućeg scenarija odstupanja u koraku 2$>$
							\item[] \begin{packed_enum}
								
								\item $<$opis rješenja mogućeg scenarija korak 1$>$
								\item $<$opis rješenja mogućeg scenarija korak 2$>$
								
							\end{packed_enum}
							\item[2.b] $<$opis mogućeg scenarija odstupanja u koraku 2$>$
							\item[3.a] $<$opis mogućeg scenarija odstupanja  u koraku 3$>$
							
						\end{packed_item}
					\end{packed_item}
				
					
				\subsubsection{Dijagrami obrazaca uporabe}
					
					\textit{Prikazati odnos aktora i obrazaca uporabe odgovarajućim UML dijagramom. Nije nužno nacrtati sve na jednom dijagramu. Modelirati po razinama apstrakcije i skupovima srodnih funkcionalnosti.}
				\eject		
				
			\subsection{Sekvencijski dijagrami}
				
				\textbf{\textit{dio 1. revizije}}\\
				
				\textit{Nacrtati sekvencijske dijagrame koji modeliraju najvažnije dijelove sustava (max. 4 dijagrama). Ukoliko postoji nedoumica oko odabira, razjasniti s asistentom. Uz svaki dijagram napisati detaljni opis dijagrama.}
				\eject
	
		\section{Ostali zahtjevi}
		
			\textbf{\textit{dio 1. revizije}}\\
		 
			 \textit{Nefunkcionalni zahtjevi i zahtjevi domene primjene dopunjuju funkcionalne zahtjeve. Oni opisuju \textbf{kako se sustav treba ponašati} i koja \textbf{ograničenja} treba poštivati (performanse, korisničko iskustvo, pouzdanost, standardi kvalitete, sigurnost...). Primjeri takvih zahtjeva u Vašem projektu mogu biti: podržani jezici korisničkog sučelja, vrijeme odziva, najveći mogući podržani broj korisnika, podržane web/mobilne platforme, razina zaštite (protokoli komunikacije, kriptiranje...)... Svaki takav zahtjev potrebno je navesti u jednoj ili dvije rečenice.}
			 
			 
			 
	